\documentclass[main.tex]{subfiles}

\begin{document}

\section{Introduction}
\indent The past few decades have seen a significant rise in the development of automated intelligent systems aiming to increase the efficiency in a number of sectors. In order to do so, they have to be able to analyse the environment and take appropriate actions. Computer vision is the field that enables in-depth scene understanding by utilizing electromagnetic radiation (light). It combines concepts from mathematics, physics, computer science and biology to identify patterns in the data and use them to make inferences about the surrounding world. In its core it strives to emulate and improve on the human visual system refined by nature through many years of evolution. Consequently, computer vision finds indispensable application in autonomous vehicles, augmented reality, security systems and numerous other areas which improve the quality of people's lives. \\
\indent Objects are the basic building block of every environment. Therefore 3D object detection is one of the most important problems that computer vision addresses. Its ultimate goal is to estimate the real world position, dimensions and orientation of a given object instance in the field of view. This effectively amalgamates object identification with pose recovery. Scene, viewpoint and object type variability combined with truncations and occlusions of objects of interest can make the task extremely challenging for contemporary methods. \\
\indent The environment itself is observed by sensors. Often the structure of a given model depends heavily on the type of data it receives as input. Therefore the choice of sensor can introduce significant differences in computation time and overall performance. Monocular camera methods have the advantage to require relatively less data compared to other approaches. In addition, the affordability of cameras and their widespread use reduce the entry barrier to science. Consequently, a lot of emphasis has been put on monocular techniques for various computer vision tasks in recent years.\\
\indent Model performance relies on efficient processing of large quantities of data produced by the sensors. Moreover, it is often intractable to manually determine an exhaustive set of relevant features and model the process of image formation fully. Therefore the algorithm has to be able to learn from the provided data. This technique has been enabled by the rapid advances in convolutional neural network frameworks which have since become a key component of every computer vision work.\\
\indent Contemporary computer vision algorithms are still inferior to the human vision system in the general case. Accompanied by the aforementioned challenges, accurate 3D object detection remains a largely unsolved problem. This motivates the search for new methods that can overcome the current obstacles in the field and reliably estimate object pose from monocular imagery.
\end{document}