\documentclass[main.tex]{subfiles}

\begin{document}
The objective of this report is to present the work carried out for my final year individual project, part of the MEng degree in Engineering Science at the University of Oxford. The project regards the recent increasingly popular topics of Robotics and Machine Learning, concentrating more specifically on certain aspects of Computer Vision, while simultaneously making use of Deep Learning and Projective Geometry. In particular, it investigates the wider task of Object Detection and proposes a novel supervised method for achieving 3D Object Pose Recovery from visual monocular imagery. While building up to this new approach, it describes methods to tackle common problems arising in these fields and presents the corresponding results. \\

Most traditional methods usually consider Object Detection and 3D Object Pose Recovery as two related but separate tasks where the quality of pose estimation often depends on the detection accuracy. Operating mostly along the same direction, the proposed approach also investigates the feasibility of performing the task without traditional object detection. At the core of the presented procedure is the task of object part segmentation. It makes use of 2D monocular visual imagery which contain multiple object instances with corresponding CAD models and viewpoints. The CAD models are split into parts and appropriate ground truth segmentation masks are then generated using projective geometry. Consequently, the segmentation task is learned with the help of convolutional neural networks. In addition, instance classification is performed to determine a matching CAD model for each detection from the initially available model set. Finally, 2D-3D pairs of key points are established combining the segmented parts with the allocated CAD model. Solving the resulting PnP-correspondence problem then allows for an approximate 3D pose to be recovered.\\

The project limits itself to the usage of the 'car' class from the PASCAL3D+ dataset in order to reduce the amount of input data and provide a feasible time frame for testing and debugging. It is believed that this could be done without significant loss of generality and that the proposed method will be readily applicable to other object classes for which CAD model and viewpoint ground truth information exists. The latter statement has to be confirmed by further research in the area.

\end{document}